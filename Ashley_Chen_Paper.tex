\documentclass[10pt, conference]{IEEEtran}

\usepackage{fancybox}
\newenvironment{boxedlaw}[1]
  {\begin{Sbox}\begin{minipage}{#1}\centering}
  {\end{minipage}\end{Sbox}\begin{center}\shadowbox{\TheSbox}\end{center}}

\ifCLASSINFOpdf
 
\else
\fi

\hyphenation{Bug Fix Verification An Analysis of Pull Request Acceptance on the Tree-Structured Level}


\begin{document}
%
% paper title
% can use linebreaks \\ within to get better formatting as desired
\title{Bug Fix Verification: An Analysis of Pull Request Acceptance on the Tree-Structured Level}


% author names and affiliations
% use a multiple column layout for up to two different
% affiliations

\author{\IEEEauthorblockN{Yu-Ann Chen}
\IEEEauthorblockA{Institute of Software Research\\Carnegie Mellon University\\yuannc@andrew.cmu.edu}
}
\maketitle


\begin{abstract}
As society becomes more dependent on Open Source Software (OSS), it becomes more important to understand how we fix bugs. Pull requests are patch proposals to OSS and are accepted if it contributes to the OSS project. But is there a way to understand the process of change and pull request acceptance in OSS? To tackle this problem, We wanted to understand what kind of structural data behavior is present in accepted pull requests, and ultimately, how this behavior can be used to classify and predict pull request acceptance. We collected pull request information from Github?s existing pull request system and processed file changes through our analysis. In the recent work around bug identification, we used what is known as naturalness and Change Distiller to mine more information about change in these files. As a result, two primary models were built, one using basic pull request information and the other using data mined from naturalness and Change Distiller. The advanced model reached an overall 64\% accuracy rate at predicting pull request acceptance, which exceeds the basic model?s accuracy rate. This approach has proven that structural analysis can be used to understand behavior of pull request acceptance. Ultimately, this work can be used to further understand what kind of bugs can be fixed automatically in OSS and how we can predict the evolution of OSS contribution.
\end{abstract}
\IEEEpeerreviewmaketitle

\section{Introduction}
Bugs have always posed a problem to the software we use. As OSS become more prevalent in society, we are faced to trusting thousands of developers around the world to build software that is dependable by major companies and governments. 

The difference between OSS and software built within a company is the scale of the project. OSS explicitly allows anyone to submit contributions and use. Closed-source software is built by a significantly smaller number of developers [add stats of ratio between contributions between OSS and CSS]. The results may generalize beyond the open source context because there is evidence that many of these practices are consistent across closed-source development organizations as well. That being said, there is also a difference in the number of bugs found in OSS and CSS [insert stats here]. This is why understanding the process of OSS is much different from CSS.

How do developers know if a bug has been fixed? In OSS, pull requests act as patch proposals and are submitted to the core developers of the repository \cite{IEEEhowto:kopka}. The only existing way to determine whether a bug has been fixed is executing the software. Ultimately, 20-25\% of pull requests are reverted [insert reference], which strongly indicates that developers are unable to fully determine whether a patch fixes a bug.

We want to answer the following research questions:

\framebox{R1. What structural data behavior is present in accepted pull requests?}\par
R2. How can this behavior be used to classify and predict pull request acceptance?
R3. What is the most useful metric for predicting pull request acceptance?

To answers these research questions, we propose an approach that builds predictive models and understands what correlation is present between tree-structural changes of accepted and rejected pull requests. This approach will be evaluated by the accuracy in prediction and the cost-effectiveness. Regardless of what results are gathered, this approach will help determine the utility of information provided on OSS service sites.

\section{Background}
In this section, we will discuss background information, which will help readers understand what we are trying to do.

\subsection{Github}
Github is a web-based version of git for developers to store projects (repositories). Developers are able to use its pull request feature to submit bugs and patches. Github is able to rank its most mature repositories by programming languages. This is evaluated by the amount of forks and stars each repository possess. 

\subsection{Naturalness}
In [insert Naturalness paper], the use of entropy and cross-entropy of a file could be used to identify buggy code. Entropy, the measure of repetitiveness and predictability is the output of analyzing source code. Cross-entropy is the measure of entropy over the code?s size (tokens).

\subsection{Change-Distiller}
Change Distiller mines fine-grained changes in source code at the tree-structured level. By taking two versions of a Java file, Change Distiller converts source code into Abstract Syntax Trees (AST). Using a tree-differencing algorithm [insert reference], the difference between the two ASTs can be extracted and classified to types and significant levels (significant level meaning the relevance of the change to the file). 

\section{Approach}
This section, we will be discussing what approach we took

\subsection{Building Models}
We used Weka, a model building application to process each ARFF file in a 10-fold cross validation. Weka distributes the data for both the training set and the testing of the model.

\subsection{Data-Analysis}
Using the model built by Weka, We create a confusion matrix to calculate precision, recall, and the AUC. This analysis can then be used to determine calculate the model?s prediction accuracy.

\subsection{Process Changes}
We used a language model called CodeMining-TreeLM to generate entropy level. Because all pull requests are from thirteen different repositories, we trained 13 models with each repositories? source code at an iteration of 400 times. Pull request?s files were matched with their respective project name, and using the serialized training set that was created by CodeMining-TreeLM, each file received an entropy and cross-entropy value.

To process files through Change Distiller, each file was given to Change Distiller in two parts: the before version and the after version of the pull request. Change Distiller then extracts changes and gives a sum of each change type from the file. 

The data mined with Naturalness and Change Distiller are saved in a text file that is specific to each pull requests. For example, a text file for Pull-Request-A may have information for 50 files because the pull request committed 50 files. The number of changes per file will be taken into account in the next step.

\subsection{Filtering Performance Metrics}
We created a script that parses through the text files of each pull request. The script selects the attributes it wants, takes the data, work out some calculation if needed, then formats it into an ARFF file which will be the input for building a model. To answer our research questions, we will be adjusting the metrics used from each database. This is why filtering the performance metrics is important to this approach.

\subsection{Data Collection}
We requested and collected the source code from each file?s rawURL using Github?s API. If a pull request was closed and merged, it would be classified as accepted. If a pull was closed and not merged, it would be classified as rejected. Our research questions only pertain to the technical and structure information of pull requests, which means we do not examine or collect information about the developers or the pull request?s comments. The pull requests we collect also can only be Java files, due to the nature of Change Distiller. Therefore, the dataset collected were also filtered to not include pull requests containing non-java files.

\section{Evaluation}
In this section, we will be reviewing the actual evaluations of the approach we took.

\subsection{Validity of Data}
The pull requests collected are validated based on the repository?s maturity. We collected these pull requests from the top 13 Java projects with the most forks, stars, and pull requests. We successfully collected 3,000 pull requests with a balanced ratio of accepted to rejected pull requests of 55 to 45.

\subsection{Predictability Accuracy Comparison}
We apply our approach to the dataset we have collected and built two models: The basic model and advanced model. The basic model encompasses only the attributes from Github, and the advanced model has a combination of the basic model?s attributes, naturalness, and structural changes. Since the basic model serves the baseline for this evaluation, the accuracy of the advanced model will be compared to the basic model. 

\subsection{Validity of Models}
To validate the classifying models used, We took five different types of classifier and measured its performance over the AUC. We took the same data from the advanced model and implemented different algorithms along with Random Forest: LMT, PART, Naive Bayes, and ZeroR. We then compare both the accuracy and the ROC Area.

\subsection{Research Questions \#1}
Not all attributes are needed in classifying pull requests. However, the naturalness of code and analyzing structural changes has shown strong correlation with accepted pull requests. 

\subsection{Research Question \#2}
We show that fine-grained source code changes in pull request can lead to improved models for predicting pull request acceptance. In Table [insert table number], the basic model received an accuracy rate of 58\% and the advanced model generated a 64\% accuracy rate. With the addition of the behavior presented in Research Question 1, the model was able to predict at a 6\% higher accuracy. Although the basic model was able to recall more accepted pulls compared to the advanced model, the advanced model had a higher precision rate of .06. Based on these results, we can answer the question: Structured behavior of accepted pull request can predict pull request acceptance by creating a model using the Random Forest algorithm.

\subsection{Research Question \# 3}
In our approach, we created various models with the same data but with a unique metric missing. Taking the accuracy of each model, we are able to measure the influence which each metric has on the overall advance model?s accuracy. In our findings, we discovered that [insert how many metrics out of total metrics] presented by the basic model only decreased the overall accuracy rate. As a result of removing these metrics, the accuracy increased from 64\% to 67\%. This could also be due to the model used with the data presented, but we argue that the quality of data has more importance in predicting accuracy than the algorithm of the model. Beyond the metrics found in the basic model, our findings also showed that the entropy metrics and structural changes had the biggest influence over the accuracy rate. The model which excluded entropy metrics had the biggest decrease in accuracy rate of 4\%. We pose that the ?usefulness? of a metric is defined by its positive influence over the overall accuracy We can now answer the third research question: Entropy and structural changes had the most influence over the accuracy of the advanced model.

\section{Conclusion}
Throughout this paper, we analyzed what change occurs in pull requests. The following are the results we have concluded:
R1. High entropy and structural changes are present in accepted pull requests.
R2. The model with the these behavior as performance metrics performed better with a higher accuracy rate.
R3. Entropy and structural changes are also the biggest influence in the model.
This paper also presents the argument that analyzing code at the structural is a better method of collecting data. As this method could be applied to other research questions, measuring the specific elements of code can present better models for prediction or classification of code. Overall, the nature of this project contributes findings to understanding the nature of social computing, specifically OSS. 

\subsection{Threats to Validity}
This project faces one particular threat to its validity. In the nature of bug fixing and the definition of a bug, a pull request acceptance may not indicate whether a pull request fixed a bug. This could alter the collected data and detour from the high-level purpose of the project. However, the advanced model proved to have a high predicting accuracy, this threat may not as be detrimental as we thought. Additional threat to validity includes limiting dataset to just Java projects. It is evident that Java projects are more explicit with its data structure than other programming languages. In terms of practicality, the approach we took may not work for the OSS written in other languages.

\subsection{Future Work}
We hope to expand this work to more programming languages. This would involve understanding Change Distiller?s algorithm and implementing its AST converter to languages like C and Python As the high-level goal is to understand the process of bug fixing in OSS, this work can be expanded to understand the behavior of the fix itself. Furthermore, if common bugs are found, there should be a common solution.

\section*{Acknowledgment}

The authors would like to thank...
more thanks here



\begin{thebibliography}{2}

\bibitem{IEEEhowto:kopka}
H.~Kopka and P.~W. Daly, \emph{A Guide to \LaTeX}, 3rd~ed.\hskip 1em plus
  0.5em minus 0.4em\relax Harlow, England: Addison-Wesley, 1999.
\bibitem{IEEEhowto:kopka}
H.~Kopka and P.~W. Daly, \emph{A Guide to \LaTeX}, 3rd~ed.\hskip 1em plus
  0.5em minus 0.4em\relax Harlow, England: Addison-Wesley, 1999.
\bibitem{IEEEhowto:kopka}
H.~Kopka and P.~W. Daly, \emph{A Guide to \LaTeX}, 3rd~ed.\hskip 1em plus
  0.5em minus 0.4em\relax Harlow, England: Addison-Wesley, 1999.
\bibitem{IEEEhowto:kopka}
H.~Kopka and P.~W. Daly, \emph{A Guide to \LaTeX}, 3rd~ed.\hskip 1em plus
  0.5em minus 0.4em\relax Harlow, England: Addison-Wesley, 1999.
\bibitem{IEEEhowto:kopka}
H.~Kopka and P.~W. Daly, \emph{A Guide to \LaTeX}, 3rd~ed.\hskip 1em plus
  0.5em minus 0.4em\relax Harlow, England: Addison-Wesley, 1999.
\bibitem{IEEEhowto:kopka}
H.~Kopka and P.~W. Daly, \emph{A Guide to \LaTeX}, 3rd~ed.\hskip 1em plus
  0.5em minus 0.4em\relax Harlow, England: Addison-Wesley, 1999.

\end{thebibliography}


% that's all folks
\end{document}


